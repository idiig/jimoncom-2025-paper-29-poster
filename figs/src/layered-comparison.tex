\documentclass[border=2pt,tikz]{standalone}

\usepackage{luatexja-fontspec}
\setmainjfont{HaranoAjiMincho-Regular}

\usepackage{color}
\usepackage{xcolor}
\usepackage{amsmath}
\usepackage{svg}
\usepackage[hidelinks]{hyperref}
\hypersetup{%
  colorlinks=true, linktoc=all, 
}

\usepackage{tikz}
\usepackage{pgfplots}
\usetikzlibrary{calc,positioning,backgrounds,fit,shapes,decorations.pathmorphing,shapes.geometric,arrows.meta,arrows,fadings,shadows}
\usepackage{pgfplotstable}

%%%%%%%%%%%%%%%%%%%%%%%%%%%%%%%%%%%%%%%%%%%%%%%%%%%%%%%%%%%%%%%%
%%% global variables
%%%%%%%%%%%%%%%%%%%%%%%%%%%%%%%%%%%%%%%%%%%%%%%%%%%%%%%%%%%%%%%%
\newcommand{\ut}{4mm}
\newcommand{\varSpacing}{6}
\newcommand{\lectSpacing}{12}
\newcommand{\unitSpacing}{3}

%%%%%%%%%%%%%%%%%%%%%%%%%%%%%%%%%%%%%%%%%%%%%%%%%%%%%%%%%%%%%%%%
%%% styles
%%%%%%%%%%%%%%%%%%%%%%%%%%%%%%%%%%%%%%%%%%%%%%%%%%%%%%%%%%%%%%%%

% ellipse style for variables
\tikzset{
  variables/.style args={#1}{
    ellipse,
    draw,
    dashed,
    minimum width=12*\ut,
    minimum height=20*\ut,
    label={[fill=white,text=black,anchor=center,align=center,font={\Huge}]above:{#1}}
  },
}

% rectangle style for lects
\tikzset{
  lects/.style args={#1}{
    rectangle,
    draw,
    minimum width=12*\ut,
    minimum height=20*\ut,
    inner sep=6em,
    label={[fill=white,text=black,anchor=center,align=center,font={\Huge}]above:{#1}}
  },
}

% circle style for variants
\tikzset{
  variants/.style args={#1}{
    draw,
    circle,
    fill=black,
    minimum height=.5*\ut,
    minimum width=.5*\ut,
    align=center,
    label={[text=black,anchor=south,align=center,font={\LARGE}]above:{#1}}
  },
}

% background box style
\tikzset{
  bk/.style args={#1}{
    draw,
    ultra thick,
    fill=#1,
    opacity=0.2,
    inner sep=1em,
    align=right,
    drop shadow={shadow xshift=0.5cm, shadow yshift=-0.5cm, opacity=0.02}
  },
}

% level label style
\tikzset{
  level/.style={
    inner sep=0pt,
    outer sep=0pt,
    font={\Huge\bfseries},
    align=right,
    xshift=-8*\ut,
    minimum width=28*\ut,
    minimum height=10*\ut
  },
}

% arrow and label styles
\tikzset{
  connection arrow/.style={
    -Latex,
    very thick
  },
  function label/.style={
    draw,
    fill=white,
    align=center,
    font={\Large}
  },
}

% pgfplots style for comparison charts
\pgfplotsset{
  ex/.style args={#1}{
    ybar,
    xtick=data,
    title={変種 #1 (Lect #1)},
    title style={font=\Huge\bfseries,anchor=north,yshift=-1em,red},
    ylabel={使用率 Usage (\%)},
    ylabel style={font=\Huge\bfseries,anchor=north},
    xlabel style={font=\Huge\bfseries,anchor=south,yshift=-1.5em},
    ymin=0,
    ymax=45,
    bar width=1.2cm,
    nodes near coords,
    enlargelimits=0.26,
    yticklabels={},
    line width=2pt,
    tick align=inside,
    tick style={line width=1pt},
    width=20*\ut,
    height=20*\ut,
    xticklabel style={font=\huge,anchor=south}
  },
}

%%%%%%%%%%%%%%%%%%%%%%%%%%%%%%%%%%%%%%%%%%%%%%%%%%%%%%%%%%%%%%%%
%%% commands
%%%%%%%%%%%%%%%%%%%%%%%%%%%%%%%%%%%%%%%%%%%%%%%%%%%%%%%%%%%%%%%%

% draw variable set (three variables in vertical arrangement)
\newcommand{\drawVariableSet}[5]{
  % #1: prefix name, #2: reference position, #3-5: three variable labels
  \node[variables=#3] (#1A) at (#2) {};
  \node[variables=#4, below=\varSpacing of #1A.south west, anchor=north west] (#1B) {};
  \node[variables=#5, below=\varSpacing of #1B.south west, anchor=north west] (#1C) {};
}

% draw unit set (three units in vertical arrangement)
\newcommand{\drawUnitSet}[5]{
  % #1: prefix name, #2: reference position, #3-5: three unit labels
  \node[variants=#3, below=4 of #2.south] (#1_alpha) {};
  \node[variants=#4, below=\unitSpacing of #1_alpha.south] (#1_beta) {};
  \node[variants=#5, below=\unitSpacing of #1_beta.south] (#1_gamma) {};
}

% draw b variants (three variants in vertical arrangement)
\newcommand{\drawBVariants}[5]{
  % #1: prefix name, #2: reference position, #3-5: three variant labels
  \node[variants=#3] (#1_2) at (#2.center) {};
  \node[variants=#4, above=1.6 of #1_2.north] (#1_1) {};
  \node[variants=#5, below=1.6 of #1_2.south] (#1_3) {};
}

% draw variable-level connection arrows
\newcommand{\drawVariableConnection}[4]{
  % #1: start, #2: end, #3: label text, #4: show label (0/1)
  \ifnum#4=1
    \draw[connection arrow] ([yshift=3.5em]#1.north east)[out=30,in=150] 
      to node[function label,midway]{#3} ([yshift=3.5em]#2.north west);
  \else
    \draw[connection arrow] ([yshift=3.5em]#1.north east)[out=30,in=150] 
      to ([yshift=3.5em]#2.north west);
  \fi
}

% draw variant-level connection arrows
\newcommand{\drawVariantConnection}[4]{
  % #1: start, #2: end, #3: label text, #4: show label (0/1)
  \ifnum#4=1
    \draw[connection arrow] (#1)[out=30,in=150] 
      to node[function label,midway]{#3} (#2);
  \else
    \draw[connection arrow] (#1)[out=20,in=160] to (#2);
  \fi
}

% draw comparison chart
\newcommand{\drawComparisonChart}[6]{
  % #1: reference position, #2: ex (A/B), #3: xlabel, #4: symbolic x coords, #5: coordinates, #6: fill color
  \node at (#1) {
    \begin{axis}[
      ex=#2, 
      xlabel=#3,
      symbolic x coords={#4}
    ]
      \addplot[draw=black, fill=#6] coordinates {#5};
    \end{axis}
  };
}

%%%%%%%%%%%%%%%%%%%%%%%%%%%%%%%%%%%%%%%%%%%%%%%%%%%%%%%%%%%%%%%%
%%% document
%%%%%%%%%%%%%%%%%%%%%%%%%%%%%%%%%%%%%%%%%%%%%%%%%%%%%%%%%%%%%%%%

\renewcommand{\rmdefault}{ptm}

\begin{document}

\begin{tikzpicture}[node distance=8*\ut]

%%%%%%%%%%%%%%%%%%%%%%%%%%%%%%%%%%%%%%%%%%%%%%%%%%%%%%%%%%%%%%%%
%%% layer 1: variable nodes
%%%%%%%%%%%%%%%%%%%%%%%%%%%%%%%%%%%%%%%%%%%%%%%%%%%%%%%%%%%%%%%%

% Lect A variables
\drawVariableSet{VarLectA}{0,0}{概念$(a)$\\Variable$(a)$}{概念$(b)$\\Variable$(b)$}{概念$(c)$\\Variable$(c)$}

% Lect B variables
\drawVariableSet{VarLectB}{$(VarLectAA.east) + (\lectSpacing*\ut, 0)$}{概念$(a)$\\Variable$(a^{\prime})$}{概念$(b^{\prime})$\\Variable$(b^{\prime})$}{概念$(d)$\\Variable$(d)$}

%%%%%%%%%%%%%%%%%%%%%%%%%%%%%%%%%%%%%%%%%%%%%%%%%%%%%%%%%%%%%%%%
%%% layer 2: unit nodes
%%%%%%%%%%%%%%%%%%%%%%%%%%%%%%%%%%%%%%%%%%%%%%%%%%%%%%%%%%%%%%%%

% Lect A unit nodes
\drawUnitSet{LectA}{VarLectAC}{言語単位$(\alpha)$\\Unit$(\alpha)$}{言語単位$(\beta)$\\Unit$(\beta)$}{言語単位$(\gamma)$\\Unit$(\gamma)$}

% Lect B unit nodes
\drawUnitSet{LectB}{VarLectBC}{言語単位$(\alpha^{\prime})$\\Unit$(\alpha^{\prime})$}{言語単位$(\beta)$\\Unit$(\beta)$}{言語単位$(\delta)$\\Unit$(\delta)$}

%%%%%%%%%%%%%%%%%%%%%%%%%%%%%%%%%%%%%%%%%%%%%%%%%%%%%%%%%%%%%%%%
%%% layer 3: b variants
%%%%%%%%%%%%%%%%%%%%%%%%%%%%%%%%%%%%%%%%%%%%%%%%%%%%%%%%%%%%%%%%

% Lect A b variants 
\drawBVariants{VarLectA_B}{VarLectAB}{$b_2$}{$b_1$}{$b_3$}

% Lect B b variants
\drawBVariants{VarLectB_B}{VarLectBB}{$b_2^{\prime}$}{$b_1^{\prime}$}{$b_4$}

%%%%%%%%%%%%%%%%%%%%%%%%%%%%%%%%%%%%%%%%%%%%%%%%%%%%%%%%%%%%%%%%
%%% layer 4: connection arrows
%%%%%%%%%%%%%%%%%%%%%%%%%%%%%%%%%%%%%%%%%%%%%%%%%%%%%%%%%%%%%%%%

% variable level connections
\drawVariableConnection{VarLectAA}{VarLectBA}{}{0}
\drawVariableConnection{VarLectAB}{VarLectBB}{$\mathbf{f}(x)$}{1}
\draw[connection arrow] ([yshift=3.5em]VarLectAC.north east)[out=-55,in=90] to++ (1.7,-10);
\draw[Latex-,very thick] ([yshift=3.5em]VarLectBC.north west)[out=235,in=90] to++ (-1.7,-10);

% unit level connections
\drawVariantConnection{LectA_alpha}{LectB_alpha}{$f(x)$}{1}
\drawVariantConnection{LectA_beta}{LectB_beta}{}{0}
\draw[connection arrow] (LectA_gamma)[out=0,in=90] to++ (3.4,-4);
\draw[Latex-,very thick] (LectB_gamma)[out=180,in=90] to++ (-3.4,-4);

% b variant level connections
\drawVariantConnection{VarLectA_B_1}{VarLectB_B_1}{$f(x)$}{1}
\drawVariantConnection{VarLectA_B_2}{VarLectB_B_2}{}{0}
\draw[connection arrow] (VarLectA_B_3)[out=0,in=90] to++ (3.4,-4);
\draw[Latex-,very thick] (VarLectB_B_3)[out=180,in=90] to++ (-3.4,-4);

%%%%%%%%%%%%%%%%%%%%%%%%%%%%%%%%%%%%%%%%%%%%%%%%%%%%%%%%%%%%%%%%
%%% layer 5: lect boxes and level backgrounds
%%%%%%%%%%%%%%%%%%%%%%%%%%%%%%%%%%%%%%%%%%%%%%%%%%%%%%%%%%%%%%%%

\begin{scope}[on background layer]
  \node[lects=変種A (Lect A), draw, fit=(VarLectAA)(LectA_gamma)] (LectABox) {};
  \node[lects=変種B (Lect B), draw, fit=(VarLectBA)(LectB_gamma)] (LectBBox) {};
\end{scope}

% level backgrounds
\coordinate (VarAAA) at ([xshift=12em,yshift=6em]VarLectBA.north east);
\node[level, left=1 of VarLectAA.south west, anchor=east] (Level1Label) 
  {概念の継承・一致\\\textit{consistant lexical variables}};
\node[bk=red, fit=(VarAAA)(Level1Label)] (Level1) {};

\coordinate (VarBBB) at ([xshift=12em,yshift=6em]VarLectBB.north east);
\node[level, left=1 of VarLectAB.south west, anchor=east] (Level2Label) 
  {概念の継承・一致\\(語形構成が変化する場合)\\\textit{consistant lexical variables}\\\textit{(when variants change)}};
\node[bk=green, fit=(VarBBB)(Level2Label)] (Level2) {};

\coordinate (VarCCC) at ([xshift=12em,yshift=6em]VarLectBC.north east);
\node[level, left=1 of VarLectAC.south west, anchor=east] (Level3Label) 
  {概念の消失と獲得\\\textit{lexical variable substitutions}};
\node[bk=blue, fit=(VarCCC)(Level3Label)] (Level3) {};

%%%%%%%%%%%%%%%%%%%%%%%%%%%%%%%%%%%%%%%%%%%%%%%%%%%%%%%%%%%%%%%%
%%% layer 6: level labels
%%%%%%%%%%%%%%%%%%%%%%%%%%%%%%%%%%%%%%%%%%%%%%%%%%%%%%%%%%%%%%%%

\node[level, font={\Huge}, left=1 of LectA_alpha.west, anchor=east] (Level4) 
  {語形の変化\\\textit{orthographic change}};
\draw[dashed] ([xshift=5em,yshift=2em]Level4.south west) --++ (90em,0);

\node[level, font={\Huge}, left=1 of LectA_beta.west, anchor=east] (Level5) 
  {語の継承・一致\\\textit{consistant words}};
\draw[dashed] ([xshift=5em,yshift=2em]Level5.south west) --++ (90em,0);

\node[level, font={\Huge}, left=1 of LectA_gamma.west, anchor=east] (Level6) 
  {語の消失と獲得\\\textit{word substitutions}};
\draw[dashed] ([xshift=5em,yshift=2em]Level6.south west) --++ (90em,0);

%%%%%%%%%%%%%%%%%%%%%%%%%%%%%%%%%%%%%%%%%%%%%%%%%%%%%%%%%%%%%%%%
%%% layer 7: comparison charts
%%%%%%%%%%%%%%%%%%%%%%%%%%%%%%%%%%%%%%%%%%%%%%%%%%%%%%%%%%%%%%%%

% variant distribution comparison charts (green)
\coordinate (ChartBase1) at ([xshift=20em,yshift=-7.065em]VarLectBB.east);

\node[below right=0 and 1.74 of ChartBase1, anchor=west] (Chart1A) {
  \begin{axis}[
    ex=A, 
    xlabel={語形 Variants},
    symbolic x coords={$b_1$,$b_2^{(\prime)}$,$b_3$,$b_4$}
  ]
    \addplot[draw=black, fill=green!20] 
      coordinates {($b_1$,40) ($b_2^{(\prime)}$,2) ($b_3$,20) ($b_4$,5)};
  \end{axis}
};

\node[right=9 of Chart1A] (Chart1B) {
  \begin{axis}[
    ex=B, 
    xlabel={語形 Variants},
    symbolic x coords={$b_1$,$b_2^{(\prime)}$,$b_3$,$b_4$}
  ]
    \addplot[draw=black, fill=green!20] 
      coordinates {($b_1$,2) ($b_2^{(\prime)}$,40) ($b_3$,2) ($b_4$,5)};
  \end{axis}
};

% variable distribution comparison charts (blue)
\node[below=12 of Chart1A] (Chart2A) {
  \begin{axis}[
    ex=A, 
    xlabel={概念 Variables},
    symbolic x coords={$(a)$,$(b^{(\prime)})$,$(c)$,$(d)$}
  ]
    \addplot[draw=black, fill=blue!20] 
      coordinates {($(a)$,2) ($(b^{(\prime)})$,40) ($(c)$,20) ($(d)$,5)};
  \end{axis}
};

\node[right=9 of Chart2A] (Chart2B) {
  \begin{axis}[
    ex=B, 
    xlabel={概念 Variables},
    symbolic x coords={$(a)$,$(b^{(\prime)})$,$(c)$,$(d)$}
  ]
    \addplot[draw=black, fill=blue!20] 
      coordinates {($(a)$,2) ($(b^{(\prime)})$,20) ($(c)$,10) ($(d)$,40)};
  \end{axis}
};

% word distribution comparison charts (red)
\node[below=12 of Chart2A] (Chart3A) {
  \begin{axis}[
    ex=A, 
    xlabel={語 Words},
    symbolic x coords={$\alpha$,$\beta^{(\prime)}$,$\gamma$,$\delta$}
  ]
    \addplot[draw=black, fill=white] 
      coordinates {($\alpha$,40) ($\beta^{(\prime)}$,10) ($\gamma$,20) ($\delta$,5)};
  \end{axis}
};

\node[right=9 of Chart3A] (Chart3B) {
  \begin{axis}[
    ex=B, 
    xlabel={語 Words},
    symbolic x coords={$\alpha$,$\beta^{(\prime)}$,$\gamma$,$\delta$}
  ]
    \addplot[draw=black, fill=white] 
      coordinates {($\alpha$,10) ($\beta^{(\prime)}$,20) ($\gamma$,40) ($\delta$,5)};
  \end{axis}
};

%%%%%%%%%%%%%%%%%%%%%%%%%%%%%%%%%%%%%%%%%%%%%%%%%%%%%%%%%%%%%%%%
%%% layer 8: comparison connections
%%%%%%%%%%%%%%%%%%%%%%%%%%%%%%%%%%%%%%%%%%%%%%%%%%%%%%%%%%%%%%%%

% variant comparison connection
\coordinate (ConnectBase1) at ([xshift=56em]VarLectBB);
\draw[<->,ultra thick] (ConnectBase1) --++ (9em,0em);
\draw[red,ultra thick] (Level2) --++ (58em,0em) node[draw,circle,fill=black]{} 
  --++ (0em,-11.5em) --++ (60em,0em) --++ (0em,22.5em);
\draw[red,ultra thick] (Level2) --++ (58em,0em) node[draw=red,circle,fill=red]{} 
  --++ (0em,13.5em) --++ (32em,0em) 
  node[align=left,anchor=west,draw=red,red] (MainTheme) {
    \fontsize{36pt}{40pt}\selectfont 
    語形分布間の比較\\
    \textit{Comparison between distribution of variants}
  };

% variable comparison connection
\coordinate (ConnectBase2) at ([xshift=56em]VarLectBC);
\draw[<->,ultra thick] (ConnectBase2) --++ (9em,0em);
\draw[ultra thick] (Level3) --++ (58em,0em) node[draw,circle,fill=black]{} 
  --++ (0em,-12.5em) --++ (60em,0em) --++ (0em,25em);
\draw[ultra thick] (Level3) --++ (58em,0em) node[draw,circle,fill=black]{} 
  --++ (0em,12.5em) --++ (40em,0em) 
  node[align=left,anchor=west] {
    \Huge 概念分布間の比較\\
    \textit{Comparison between distribution of variables}
  };

% word comparison connection
\coordinate (ConnectBase3) at ([xshift=21.5em,yshift=-2em]LectB_beta);
\draw[<->,ultra thick] ([xshift=34.5em]ConnectBase3) --++ (9em,0em);
\draw[ultra thick] ([xshift=-8.8em]ConnectBase3) --++ (19.2em,0em) 
  node[draw,circle,fill=black]{} --++ (0em,-12.5em) --++ (60em,0em) --++ (0em,25em);
\draw[ultra thick] ([xshift=-8.8em]ConnectBase3) --++ (19.2em,0em) 
  node[draw,circle,fill=black]{} --++ (0em,12.5em) --++ (40em,0em) 
  node[align=left,anchor=west] {
    \Huge 語の分布の比較\\
    \textit{Comparison between distribution of words}
  };

%%%%%%%%%%%%%%%%%%%%%%%%%%%%%%%%%%%%%%%%%%%%%%%%%%%%%%%%%%%%%%%%
%%% layer 9: research focus
%%%%%%%%%%%%%%%%%%%%%%%%%%%%%%%%%%%%%%%%%%%%%%%%%%%%%%%%%%%%%%%%

\draw[red,Latex-,ultra thick] ([xshift=-2em]MainTheme.north east) 
  --++ (0em,30em) --++ (-35em,0em) 
  node[red,font={\Huge\bfseries},align=right,anchor=east] (ResearchFocus) {
    \fontsize{45pt}{50pt}\selectfont 
    本研究\\
    \textit{This research}
  };

\node[below=1 of ResearchFocus.south east, align=right, 
      font={\fontsize{36pt}{40pt}\selectfont}] {
  結果の例:\\
  (Result example)\\
  $b_1$ is a Lect-A-featured unit\\
  $b_2$/$b_2^{\prime}$ is a Lect-B-featured unit
};

\end{tikzpicture}

\end{document}
