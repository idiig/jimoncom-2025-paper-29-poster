% SPDX-License-Identifier: CC-BY-SA-3.0
% SPDX-FileCopyrightText: 2025 idiig
\documentclass[border=2pt]{standalone}

\usepackage{luatexja-fontspec}
\setmainfont{HaranoAjiMincho-Regular.otf}
\setsansfont{HaranoAjiGothic-Regular.otf}
\usepackage{svg}

\begin{document}
\begin{minipage}{\textwidth}
  \begin{minipage}{\textwidth}
    \begin{minipage}{\textwidth} \includesvg[width=\textwidth]{../figs/fig-diff-phase-1}\\[-46em]
      {\hspace{6em}\color{red}\huge\bfseries ランダム水準:4コンストラクト}\\[15em]
      {\hspace{6em}\color{red}\huge\bfseries 同概念水準:13コンストラクト}\\[12em]
      {\hspace{6em}\color{red}\huge\bfseries 類義語水準:5コンストラクト}\\[12em]
    \end{minipage}\\[-53.6em]
    {\color{blue}\huge\bfseries --------------------------------------------------------}\\[10.6em]
    {\color{blue}\huge\bfseries --------------------------------------------------------}\\[4.3em]
    {\color{blue}\huge\bfseries --------------------------------------------------------}\\[10.6em]
    {\color{blue}\huge\bfseries --------------------------------------------------------}\\[4.3em]
    {\color{blue}\huge\bfseries --------------------------------------------------------}\\[10.6em]
    {\color{blue}\huge\bfseries --------------------------------------------------------}\\[5em]
  \end{minipage}\\[-2em]
  \begin{minipage}{\textwidth}
    \color{blue}\Huge\bfseries 三水準共通点:
    \begin{itemize}
    \item 「金葉→詞花」「拾遺→後拾遺」の変化が「古今→後撰」の変化より大きい
    \end{itemize}
  \end{minipage}
\end{minipage}
\end{document}
