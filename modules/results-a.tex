% SPDX-License-Identifier: CC-BY-SA-3.0
% SPDX-FileCopyrightText: 2025 idiig
\documentclass[border=2pt]{standalone}

\usepackage{luatexja-fontspec}
\setmainfont{HaranoAjiMincho-Regular.otf}
\setsansfont{HaranoAjiGothic-Regular.otf}
\usepackage{svg}

\begin{document}
\begin{minipage}{1.8\textwidth}
  \begin{minipage}{.5\textwidth}
    \includesvg[width=\textwidth]{../figs/aggregate-path}
  \end{minipage}%
  \scalebox{0.5}{%
    \begin{minipage}{.5\textwidth}
      \begin{itemize}
      \item 異なる水準のパターンに\underline{大きな相違が見られず}
      \item 後拾遺で\textbf{正負転換}(0値横断)
      \item クラスタリング:
        \begin{itemize}
        \item[◎] 「古今・後撰・拾遺」同クラスタ
        \item[◎] 「後拾遺・詞花・千載」同クラスタ
        \item[●] 「新古今」「金葉」は\textbf{中間水準で不一致}
        \end{itemize}
      \item 金葉以降の\textbf{往還}:
        \begin{itemize}
        \item[▽] 金葉(同時代重視)
        \item[▼] 詞花(後拾遺集歌人重視)
        \item[▽] 千載(同時代重視)
        \item[▼] 新古今(本歌取り隆盛)
        \end{itemize}
      \end{itemize}
    \end{minipage}%
  }%
\end{minipage}
\end{document}
