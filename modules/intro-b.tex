% SPDX-License-Identifier: CC-BY-SA-3.0
% SPDX-FileCopyrightText: 2025 idiig
\documentclass[border=2pt]{standalone}

\usepackage{luatexja-fontspec}
\setmainfont{HaranoAjiMincho-Regular.otf}
\setsansfont{HaranoAjiGothic-Regular.otf}
\usepackage{svg}
\usepackage{adjustbox}
\usepackage{xcolor}
\usepackage{tikz}
\usetikzlibrary{backgrounds,shadings}

\begin{document}
\begin{minipage}{\textwidth}
  \begin{minipage}{\textwidth}
    \includesvg[height=.25\textwidth]{../figs/simpson-paradox-1}%
    \hspace{1em}%
    \includesvg[height=.25\textwidth]{../figs/simpson-paradox-2}\\[-9em]
    \quad{\Huge\bfseries\color{gray} Simpson's paradox}
  \end{minipage}\\[6em]
  \tikz[baseline=(content.east)]{
    \node[inner sep=0pt] (content) {%
      \begin{minipage}{\textwidth}
        \Huge\bfseries $\Rightarrow$\quad
        \begin{minipage}{0.5\textwidth}\Large\bfseries Within-profile(group) Comparison\end{minipage}
      \end{minipage}%
    };
    \begin{scope}[on background layer]
      \shade[inner color=pink!90,outer color=white] ([xshift=-2cm]content.center) ellipse (4.5cm and 1cm);
    \end{scope}
  }
\end{minipage}
% \begin{minipage}{.9\textwidth}
%   \begin{itemize}
%   \item[] 2資料の語彙差は,資料のテーマ・文化もろもろを含めた差
%   \item[→] 焦点の絞り方
%   \end{itemize}
% \end{minipage}
\end{document}
